\documentclass[conference]{IEEEtran}
% \IEEEoverridecommandlockouts
% The preceding line is only needed to identify funding in the first footnote. If that is unneeded, please comment it out.
\usepackage{cite}
\usepackage{amsmath,amssymb,amsfonts}
\usepackage{algorithmic}
\usepackage{graphicx}
\usepackage{textcomp}
\usepackage{xcolor}
\def\BibTeX{{\rm B\kern-.05em{\sc i\kern-.025em b}\kern-.08em
    T\kern-.1667em\lower.7ex\hbox{E}\kern-.125emX}}
\begin{document}

\title{Investigating the Effect of Dataset Representivity for Image Colourisation}

\author{\IEEEauthorblockN{Andrew Boyley}
\IEEEauthorblockA{\textit{School of Computer Science} \\
\textit{and Applied Mathematics}\\
\textit{University of the Witwatersrand}\\
Johannesburg, South Africa \\
andrew.boyley@students.wits.ac.za}
\and
\IEEEauthorblockN{William Hill} 
\IEEEauthorblockA{\textit{School of Computer Science} \\
\textit{and Applied Mathematics}\\
\textit{University of the Witwatersrand}\\
Johannesburg, South Africa \\
william.hill1@students.wits.ac.za}
\and
\IEEEauthorblockN{Steven James}
\IEEEauthorblockA{\textit{School of Computer Science} \\
\textit{and Applied Mathematics}\\
\textit{University of the Witwatersrand}\\
Johannesburg, South Africa \\
steven.james@wits.ac.za}
}

\maketitle

\begin{abstract}

\end{abstract}

\begin{IEEEkeywords}
deep learning, bias, dataset imbalance, regression, self-supervised learning
\end{IEEEkeywords}

\section{Introduction}

Machine Learning is a tentative endeavour, whereby subtle changes in the learning environment often lead to profound alterations to both the overall effectiveness of the machine learning model and the accuracy of the desired output. 

Such a phenomenon exists when training a model to recognise and colourise aspects of grayscale images: by employing a training dataset with a bias to a particular locale – like an American or a Eurocentric setting, for instance – any sample taken from outside such an area (with particular focus placed on a South African sample) is misrepresented and yields erroneous results, with the produced colourised image being quintessentially different to the intended image.

The identified cause is the lack of diversity in the training datasets because even though various methods can be implemented in the actual model to mitigate this bias, they cannot account for a wholly prejudiced dataset being used initially, particularly a non-South African dataset being used to train a model to operate on a South African sample. 

Therefore, the aim is to investigate what percentage of the training dataset has to be of local origin compared to the international origin of the remainder of the dataset. However, constructing and testing datasets to determine such a goal is both time consuming and expensive in terms of financial and computational resources required. Hence, the approach of image colourisation has been selected as it allows an uncomplicated procedure of collecting publicly-available images of both International and South African settings, thus providing a simple way to build a diverse dataset which does not need to be labelled or require excessive computational time since the machine learning approach required is one of supervised learning where the expected outcome is derived from the images themselves.

Consequently, the refined aim is to investigate the optimal ratio of non-South African images to South African images in the training dataset is such that bias is minimised and the model performs optimally for both sets of images.

\section{Background}

Explain notation and setup for regression/classification task in general i.e. objective function

Describe LAB; classes, etc. 

Describe eval criteria

Describe neural network, backprop, Insert image if space


\section{Proposed Method}

Train colouriser on dataset X. Evaluate accuracy

Explain how our dataset is built precisely: using flickr data, keywords, any preprocessing etc. 

Show some images. 

Maybe even compute mean statistics of dataset X and Y to show that they are in fact different?

Explain method - compute accuracy on X -> X.
Then check X -> Y

Then check how much of Y must be added to X to improve things?

Would be nice to have multiple datasets, models, but maybe no time :(

\section{Experiments}

Explain experiment. Model used. Hyperparameters. Early stopping?? Train/val/test splits.

Show training curves over time for model trained on X. 

Show images from X coloured.
Show images from Y coloured - qualitative analysis

Start adding data from Y to X. 

Produce THE curve as a function of data ratio

Show images from X coloured.
Show images from Y coloured - qualitative analysis



\section{Related Work}

\begin{itemize}
    \item Find literature on dataset imbalance in general
    \item Find literature on racial bias in general
\end{itemize}

\section{Conclusion}

Important problem

Investigated data representivity

What we found

Extensions to other tasks.


\bibliographystyle{IEEEtran}
\bibliography{IEEEfull,colourisation}

% \begin{thebibliography}{00}
% \bibitem{b1} G. Eason, B. Noble, and I. N. Sneddon, ``On certain integrals of Lipschitz-Hankel type involving products of Bessel functions,'' Phil. Trans. Roy. Soc. London, vol. A247, pp. 529--551, April 1955.
% \bibitem{b2} J. Clerk Maxwell, A Treatise on Electricity and Magnetism, 3rd ed., vol. 2. Oxford: Clarendon, 1892, pp.68--73.
% \bibitem{b3} I. S. Jacobs and C. P. Bean, ``Fine particles, thin films and exchange anisotropy,'' in Magnetism, vol. III, G. T. Rado and H. Suhl, Eds. New York: Academic, 1963, pp. 271--350.
% \bibitem{b4} K. Elissa, ``Title of paper if known,'' unpublished.
% \bibitem{b5} R. Nicole, ``Title of paper with only first word capitalized,'' J. Name Stand. Abbrev., in press.
% \bibitem{b6} Y. Yorozu, M. Hirano, K. Oka, and Y. Tagawa, ``Electron spectroscopy studies on magneto-optical media and plastic substrate interface,'' IEEE Transl. J. Magn. Japan, vol. 2, pp. 740--741, August 1987 [Digests 9th Annual Conf. Magnetics Japan, p. 301, 1982].
% \bibitem{b7} M. Young, The Technical Writer's Handbook. Mill Valley, CA: University Science, 1989.
% \end{thebibliography}

\end{document}
