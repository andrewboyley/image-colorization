\documentclass[conference]{IEEEtran}
% \IEEEoverridecommandlockouts
% The preceding line is only needed to identify funding in the first footnote. If that is unneeded, please comment it out.
\usepackage{cite}
\usepackage{amsmath,amssymb,amsfonts}
\usepackage{algorithmic}
\usepackage{graphicx}
\usepackage{textcomp}
\usepackage{xcolor}
\def\BibTeX{{\rm B\kern-.05em{\sc i\kern-.025em b}\kern-.08em
    T\kern-.1667em\lower.7ex\hbox{E}\kern-.125emX}}
\begin{document}

\title{Investigating the Effect of Dataset Representivity for Image Colourisation}

\author{\IEEEauthorblockN{Andrew Boyley}
\IEEEauthorblockA{\textit{School of Computer Science} \\
\textit{and Applied Mathematics}\\
\textit{University of the Witwatersrand}\\
Johannesburg, South Africa \\
andrew.boyley@students.wits.ac.za}
\and
\IEEEauthorblockN{William Hill} 
\IEEEauthorblockA{\textit{School of Computer Science} \\
\textit{and Applied Mathematics}\\
\textit{University of the Witwatersrand}\\
Johannesburg, South Africa \\
william.hill1@students.wits.ac.za}
\and
\IEEEauthorblockN{Steven James}
\IEEEauthorblockA{\textit{School of Computer Science} \\
\textit{and Applied Mathematics}\\
\textit{University of the Witwatersrand}\\
Johannesburg, South Africa \\
steven.james@wits.ac.za}
}

\maketitle

\begin{abstract}

\end{abstract}

\begin{IEEEkeywords}
deep learning, bias, dataset imbalance, regression, self-supervised learning
\end{IEEEkeywords}

\section{Introduction}

Well known that bias in ML is a problem. 

Due to lack of diversity in training examples

Some work on how to mitigate these issues in training, but what about the data itself?

We investigate how much data we need to make the problem go away. 

Expensive to construct datasets. Need time, money, info.
We therefore select the task of image colourisation, since it is a supervised approach. We need no labelling, only data, since the regression targets are derived from the images themselves!
We get this data from publically accessible sources captured from variety of places across SA/Africa. 
Easy to build a diverse dataset in this way. 

We train image colourisation on dataset X, and test on our dataset Y.  I assume bad things happpen.

We look at how much data we need to augment X from Y to fix the problem. We find (insert results) 

\section{Background}

Explain notation and setup for regression/classification task in general i.e. objective function

Describe LAB; classes, etc. 

Describe eval criteria

Describe neural network, backprop, Insert image if space


\section{Proposed Method}

Train colouriser on dataset X. Evaluate accuracy

Explain how our dataset is built precisely: using flickr data, keywords, any preprocessing etc. 

Show some images. 

Maybe even compute mean statistics of dataset X and Y to show that they are in fact different?

Explain method - compute accuracy on X -> X.
Then check X -> Y

Then check how much of Y must be added to X to improve things?

Would be nice to have multiple datasets, models, but maybe no time :(

\section{Experiments}

Explain experiment. Model used. Hyperparameters. Early stopping?? Train/val/test splits.

Show training curves over time for model trained on X. 

Show images from X coloured.
Show images from Y coloured - qualitative analysis

Start adding data from Y to X. 

Produce THE curve as a function of data ratio

Show images from X coloured.
Show images from Y coloured - qualitative analysis



\section{Related Work}

\begin{itemize}
    \item Find literature on dataset imbalance in general
    \item Find literature on racial bias in general
\end{itemize}

\section{Conclusion}

Important problem

Investigated data representivity

What we found

Extensions to other tasks.


\bibliographystyle{IEEEtran}
\bibliography{IEEEfull,colourisation}

% \begin{thebibliography}{00}
% \bibitem{b1} G. Eason, B. Noble, and I. N. Sneddon, ``On certain integrals of Lipschitz-Hankel type involving products of Bessel functions,'' Phil. Trans. Roy. Soc. London, vol. A247, pp. 529--551, April 1955.
% \bibitem{b2} J. Clerk Maxwell, A Treatise on Electricity and Magnetism, 3rd ed., vol. 2. Oxford: Clarendon, 1892, pp.68--73.
% \bibitem{b3} I. S. Jacobs and C. P. Bean, ``Fine particles, thin films and exchange anisotropy,'' in Magnetism, vol. III, G. T. Rado and H. Suhl, Eds. New York: Academic, 1963, pp. 271--350.
% \bibitem{b4} K. Elissa, ``Title of paper if known,'' unpublished.
% \bibitem{b5} R. Nicole, ``Title of paper with only first word capitalized,'' J. Name Stand. Abbrev., in press.
% \bibitem{b6} Y. Yorozu, M. Hirano, K. Oka, and Y. Tagawa, ``Electron spectroscopy studies on magneto-optical media and plastic substrate interface,'' IEEE Transl. J. Magn. Japan, vol. 2, pp. 740--741, August 1987 [Digests 9th Annual Conf. Magnetics Japan, p. 301, 1982].
% \bibitem{b7} M. Young, The Technical Writer's Handbook. Mill Valley, CA: University Science, 1989.
% \end{thebibliography}

\end{document}
